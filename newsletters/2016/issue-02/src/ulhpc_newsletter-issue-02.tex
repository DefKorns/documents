%=============================================================================
% File:  ulhpc_newsletter-issue-02.tex --  Issue 2 of the UL HPC newsletter
% Time-stamp: <Thu 2016-01-28 15:58 svarrette>
% Author(s): Sebastien Varrette <Sebastien.Varrette@uni.lu>
% .
%=============================================================================
\documentclass{article}

\usepackage{graphicx}
\graphicspath{{logos/}{images/}}

%%%%%%%%%%%%%%% UL HPC Newsletter Configuration %%%%%%%%%%%%%%%%%%%
\def\ReleaseDate{January 2016} % Release
\def\Number{2}                  % Issue Number
\def\Editorsname{Dr. S. Varrette and Prof. P. Bouvry}
\def\DocSrcURL{https://github.com/ULHPC/documents/tree/master/newsletters/2016/issue-02/}
\def\NewsletterURL{http://hpc.uni.lu/newsletters}
\usepackage{ulhpc_newsletter1}
\usepackage{eurosym}
\usepackage{multirow}
\bibliographystyle{plain}




\newcommand{\RUES}{\href{http://wwwfr.uni.lu/research/fstc/research_unit_in_engineering_science_rues}{RUES}}
\newcommand{\LCSB}{\href{http://lcsb.uni.lu}{LCSB}}
\newcommand{\PHYMS}{\href{http://wwwen.uni.lu/research/fstc/physics_and_materials_science_research_unit}{PHYMS}}
\newcommand{\SnT}{\href{http://snt.uni.lu}{SnT}}
\newcommand{\LSF}{\href{http://lsf.uni.lu}{LSF}}
\newcommand{\Haswell}{\href{https://en.wikipedia.org/wiki/Haswell_(microarchitecture)}{Haswell}}
\begin{document}

% Format: \leftinfo{<title>}{<body>}
\ulhpctoc

% Format: \begin{welcome}[<title>] ... \end{welcome}
\begin{welcome}
  \input{_welcome.md}
\end{welcome}


%=======================================================
\section{\faCommentAlt\ UL HPC at a glance}
%=======================================================
\input{_ul_hpc.md}
\clearpage

%=======================================================
\section{\faBarChart\ Platform Statistics}
%=======================================================
\input{_stats.md}
% =============================================================================
% File:  _table_HPC_platform_around.tex --
% Author(s): Sebastien Varrette <Sebastien.Varrette@uni.lu>
% Creation:  27 Jan 2016
% Time-stamp: <Thu 2016-01-28 10:47 svarrette>
% 
% Copyright (c) 2016 Sebastien Varrette <Sebastien.Varrette@uni.lu>
% 
% $Id$
% 
% More information on LaTeX: http://www.latex-project.org/
% LaTeX symbol list:
% http://www.ctan.org/tex-archive/info/symbols/comprehensive/symbols-a4.pdf
% =============================================================================

% Reference: see https://docs.google.com/spreadsheets/d/1FhLLoFCNn2G3xuiITl8dYoB3b_X6HDDzxBv0iBDcmns

%== Luxembourg
\newcommand{\UL}{\href{https://hpc.uni.lu/}{\textbf{UL HPC} (Uni.lu)}}
\newcommand{\LIST}{\href{http://www.clustervision.com/content/case-studies/CRPGL}{LIST}}
%== France
\newcommand{\LORIA}{\href{https://www.grid5000.fr/mediawiki/index.php/Nancy:Hardware\#Clusters}{LORIA (G5K), Nancy}}
\newcommand{\ROMEO}{\href{https://romeo.univ-reims.fr/pages/presentation_hardware}{ROMEO, Reims}}
%== Belgium 
\newcommand{\ULiege}{\href{http://www.ulg.ac.be/cms/c_3826073/en/nic4-detailed-explanation}{NIC4, University of Liège}}
\newcommand{\UCL}   {\href{http://www.cism.ucl.ac.be/faq/}{Université Catholique de Louvain}}
\newcommand{\UGent} {\href{https://www.vscentrum.be/infrastructure/hardware/hardware-ugent}{UGent / VSC, Gent}}
%== Germany
\newcommand{\bwGrid}      {\href{http://www.urz.uni-heidelberg.de/server/grid/hardware.en.html}{bwGrid, Heidelberg}}
\newcommand{\bwForCluster}{\href{http://www.bwhpc-c5.de/wiki/index.php/Hardware_and_Architecture_(bwForCluster_Chemistry)}{bwForCluster, Ulm}}
\newcommand{\bwHPC}       {\href{https://www.bwhpc-c5.de/wiki/index.php/Hardware_and_Architecture_(bwForCluster_MLS\%26WISO)}{bwHPC MLS\&WISO, Mannheim}}

%_______________
\begin{table}[h]
  \centering\scriptsize
  \begin{tabular}{|l|l||l|l|l|l|l|}
    \multicolumn{3}{c}{} & \multicolumn{1}{c}{(CPU)} & \multicolumn{1}{c}{TFlops} & \multicolumn{1}{c}{TB (Shared)} \\
    \hline
    \rowcolor{lightgray}
    \textbf{Country} & \textbf{Institute} & \textbf{\#Nodes} & \textbf{\#Cores} & \textbf{R$_{peak}$} & \textbf{Storage} \\\hline
    \hline
    % ___________________________ <Name>        <nodes>   <cores>       <TFlops>      <TB>
    \multirow{2}{*}{Luxembourg} & \UL  & \textbf{518} & \textbf{5316} & \textbf{87} & \textbf{5114} \\
                         &       \LIST         &  58     &  800        & 6.21         & 144  \\\hline
    \hline
    \multirow{2}{*}{France} &    \LORIA        & 320     & 2520        & 26.98        & 82   \\   
                         &       \ROMEO        & 174     & 3136        & 49.26        & 245  \\\hline
    \hline
    \multirow{3}{*}{Belgium} &   \ULiege       & 128     & 2048        & 32.00        & 20   \\
                         &       \UCL          & 112     & 1344        & 13.28        & 120  \\
                         &       \UGent        & 440     & 8768        & 275.30       & 1122 \\\hline
    \hline
    \multirow{3}{*}{Germany} &   \bwGrid       & 140     & 1120        & 12.38        & 32   \\
                         &       \bwForCluster & 444     & 7104        & 266.40       & 400  \\
                         &       \bwHPC        & 604     & 9728        & 371.60       & 420  \\
    \hline  
  \end{tabular}
\end{table}


% ~~~~~~~~~~~~~~~~~~~~~~~~~~~~~~~~~~~~~~~~~~~~~~~~~~~~~~~~~~~~~~~~
% eof
% 
% Local Variables:
% mode: latex
% mode: flyspell
% mode: visual-line
% TeX-master: "ulhpc_newsletter-issue-02"
% End:



% Similar to an executive summary of (incoming/passed)
% milestones (half page)

%===========================================================
\section{\faBullhorn\ What happened in 2015?}
%===========================================================
%\vspace*{-1em}
\input{_structure.md}
\input{_new_hardware.md}
\clearpage
\input{_gpfs.md}
\input{_os_upgrade.md}
\input{_new_services.md}


% %\newpage

%=======================================================
\section{\faTime\ What's next?}
%=======================================================
\input{_CDC.md}
\input{_IPCEI.md}
% \clearpage





% New features: services/improvements we've introduced, new software
% available, etc.

% %=======================================================
% \section{\faBarChart\ Latest Statistics}
% %=======================================================

% Platform overview (pretty graphs): for the passing \$period (maybe also
% with last year history): \#jobs , data transferred,  new users/projects,
% etc., also reminder of existing services (gitlab, gforge, owncloud, etc.)

% %=======================================================
% \section{\faShare\ Outcomes}
% %=======================================================

% Outcomes: projects/groups using the clusters, new articles published
% referencing the platform, etc.

% %=======================================================
% \section{\faUser\ User Focus}
% %=======================================================

% User stories: 1-2 paragraphs from (1-2) 'big' users, describing what
% they're doing

%\newpage
% %=======================================================
\section{\faCalendar\ 2015 HPC Schools}
% %=======================================================
% Upcoming events: new tutorials, software we'll introduce in the next
% \$period, upcoming HPC schools etc.
\input{_hpc_school.md}


% %=======================================================
\section{\faFileAlt\ Publications}
% %=======================================================
\input{_publis.md}



% %=======================================================
\clearpage
\section{\faUser\ Meet the team}
% %=======================================================
% Upcoming events: new tutorials, software we'll introduce in the next
% \$period, upcoming HPC schools etc.
\input{_hpc_team.md}

% %============================================================================================
% \section{\faCogs\ How to use the \texttt{ulhpc\_newsletter1} style?}
% %============================================================================================

% \faWarningSign\ : You need to compile the newsletters with \texttt{lualatex}!

% If you are using \href{https://github.com/Falkor/Makefiles/blob/devel/latex/Makefile}{my generic Makefile}, simply place a file \texttt{.Makefile.local} at the root of your document having the following content:

% \lstinputlisting[language=make]{.Makefile.local}

% Then in addition to the addition of the \LaTeX\ style file \texttt{ulhpc\_newsletter1.sty} at the root of your document, you'll need:

% \begin{itemize}
%   \item a copy of the logos files (in PDF format) \texttt{logo\_UL.pdf} and \texttt{logo\_ULHPC.pdf} in a know path;
%   \item to initiate a certain number of variables / macros used in the style \textbf{prior to} the call to the style package.
% \end{itemize}

% They are defined as follows:

% \lstinputlisting[language=TeX,firstline=45,lastline=70]{ulhpc_newsletter1.sty}


% % You need first to clone (directly or as a git submodule) the \texttt{ULHPC/documents} repository on \href{https://github.com/ULHPC/documents}{Github}.

% % Then in addition to the addition of the \LaTeX\ style file \texttt{ulhpc\_book.sty} at the root of your document, you'll need:

% %=======================================================
% \section{\faBeer\ Last but not least}
% %=======================================================
% \leftinfo{Another highlight}{For this page... }

% \lipsum[2-8]

%\bibliography{biblio}

\end{document}



%~~~~~~~~~~~~~~~~~~~~~~~~~~~~~~~~~~~~~~~~~~~~~~~~~~~~~~~~~~~~~~~~
% eof
%
% Local Variables:
% mode: latex
% mode: flyspell
% mode: visual-line
% TeX-master: "ulhpc_newsletter-issue-02"
% End:
