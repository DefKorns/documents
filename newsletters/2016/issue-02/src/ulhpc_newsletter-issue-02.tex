%=============================================================================
% File:  ulhpc_newsletter-issue-02.tex --  Issue 2 of the UL HPC newsletter
% Time-stamp: <Wed 2016-01-27 16:45 svarrette>
% Author(s): Sebastien Varrette <Sebastien.Varrette@uni.lu>
% .
%=============================================================================
\documentclass{article}

\usepackage{graphicx}
\graphicspath{{logos/}{images/}}

%%%%%%%%%%%%%%% UL HPC Newsletter Configuration %%%%%%%%%%%%%%%%%%%
\def\ReleaseDate{January 2016} % Release
\def\Number{2}                  % Issue Number
\def\Editorsname{Dr. S. Varrette and Prof. P. Bouvry}
\def\DocSrcURL{https://github.com/ULHPC/documents/tree/master/newsletters/2015/issue-02/}
\def\NewsletterURL{http://hpc.uni.lu/newsletters}
\usepackage{ulhpc_newsletter1}
\usepackage{eurosym}

\bibliographystyle{plain}

\begin{document}

% Format: \leftinfo{<title>}{<body>}
\ulhpctoc

% Format: \begin{welcome}[<title>] ... \end{welcome}
\begin{welcome}
  \input{_welcome.md}
\end{welcome}


%=======================================================
\section{\faCommentAlt\ UL HPC at a glance}
%=======================================================
\input{_ul_hpc.md}
\clearpage

%=======================================================
\section{\faBarChart\ Platform Statistics}
%=======================================================
\input{_stats.md}
%\clearpage


% Similar to an executive summary of (incoming/passed)
% milestones (half page)


%=======================================================
\section{\faBullhorn\ What's new?}
%=======================================================
\input{_structure.md}

%\clearpage
\input{_os_upgrade.md}
\input{_gpfs.md}
\clearpage
\input{_new_services.md}
\input{_new_hardware.md}

%\newpage

%=======================================================
\section{\faTime\ What's next?}
%=======================================================
\input{_CDC.md}

\clearpage





% New features: services/improvements we've introduced, new software
% available, etc.

% %=======================================================
% \section{\faBarChart\ Latest Statistics}
% %=======================================================

% Platform overview (pretty graphs): for the passing \$period (maybe also
% with last year history): \#jobs , data transferred,  new users/projects,
% etc., also reminder of existing services (gitlab, gforge, owncloud, etc.)

% %=======================================================
% \section{\faShare\ Outcomes}
% %=======================================================

% Outcomes: projects/groups using the clusters, new articles published
% referencing the platform, etc.

% %=======================================================
% \section{\faUser\ User Focus}
% %=======================================================

% User stories: 1-2 paragraphs from (1-2) 'big' users, describing what
% they're doing

%\newpage
% %=======================================================
\section{\faCalendar\ 2015 HPC Schools}
% %=======================================================
% Upcoming events: new tutorials, software we'll introduce in the next
% \$period, upcoming HPC schools etc.
\input{_hpc_school.md}


% %=======================================================
\section{\faFileAlt\ Publications}
% %=======================================================
\input{_publis.md}



% %=======================================================
\clearpage
\section{\faUser\ Meet the team}
% %=======================================================
% Upcoming events: new tutorials, software we'll introduce in the next
% \$period, upcoming HPC schools etc.
\input{_hpc_team.md}

% %============================================================================================
% \section{\faCogs\ How to use the \texttt{ulhpc\_newsletter1} style?}
% %============================================================================================

% \faWarningSign\ : You need to compile the newsletters with \texttt{lualatex}!

% If you are using \href{https://github.com/Falkor/Makefiles/blob/devel/latex/Makefile}{my generic Makefile}, simply place a file \texttt{.Makefile.local} at the root of your document having the following content:

% \lstinputlisting[language=make]{.Makefile.local}

% Then in addition to the addition of the \LaTeX\ style file \texttt{ulhpc\_newsletter1.sty} at the root of your document, you'll need:

% \begin{itemize}
%   \item a copy of the logos files (in PDF format) \texttt{logo\_UL.pdf} and \texttt{logo\_ULHPC.pdf} in a know path;
%   \item to initiate a certain number of variables / macros used in the style \textbf{prior to} the call to the style package.
% \end{itemize}

% They are defined as follows:

% \lstinputlisting[language=TeX,firstline=45,lastline=70]{ulhpc_newsletter1.sty}


% % You need first to clone (directly or as a git submodule) the \texttt{ULHPC/documents} repository on \href{https://github.com/ULHPC/documents}{Github}.

% % Then in addition to the addition of the \LaTeX\ style file \texttt{ulhpc\_book.sty} at the root of your document, you'll need:

% %=======================================================
% \section{\faBeer\ Last but not least}
% %=======================================================
% \leftinfo{Another highlight}{For this page... }

% \lipsum[2-8]

%\bibliography{biblio}

\end{document}



%~~~~~~~~~~~~~~~~~~~~~~~~~~~~~~~~~~~~~~~~~~~~~~~~~~~~~~~~~~~~~~~~
% eof
%
% Local Variables:
% mode: latex
% mode: flyspell
% mode: visual-line
% TeX-master: "ulhpc_newsletter-issue-02"
% End:
