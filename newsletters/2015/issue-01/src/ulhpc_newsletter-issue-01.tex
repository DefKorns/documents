%=============================================================================
% File:  ulhpc_newsletter-issue-01.tex --  Issue 1 of the UL HPC newsletter
% Time-stamp: <Jeu 2015-03-05 10:56 svarrette>
% Author(s): Sebastien Varrette <Sebastien.Varrette@uni.lu>
% .          
%=============================================================================
\documentclass{article}

\usepackage{graphicx}
\graphicspath{{logos/}{images/}}

%%%%%%%%%%%%%%% UL HPC Newsletter Configuration %%%%%%%%%%%%%%%%%%%
\def\ReleaseDate{February 2015} % Release
\def\Number{1}                  % Issue Number
\def\Editorsname{Dr. S. Varrette and Prof. P. Bouvry}
\def\DocSrcURL{https://github.com/ULHPC/documents/tree/master/newsletters/2015/issue-01/}
\def\NewsletterURL{http://hpc.uni.lu/newsletters}
\usepackage{ulhpc_newsletter1}

\begin{document}

% Format: \leftinfo{<title>}{<body>}
\ulhpctoc

% Format: \begin{welcome}[<title>] ... \end{welcome}
\begin{welcome}
  \input{_welcome.md}
\end{welcome}

   


%   As part of our commitment to constantly improve the services offered by the
% HPC
% platform, we are announcing that we will proceed with an upgrade of the
% underlying software running the platform during Q1 2015.

% In effect, we will migrate the OS of the computing nodes from Debian 6
% (Squeeze)
% to Debian 7 (Wheezy). The software environment available through the Modules
% system will also be regenerated, the technical details are described below.
% The storage on Gaia will be reorganized and migrated for better stability.

% As it is our wish to make the upgrades as seamless as possible for you, we
% will
% perform them in several phases, allowing for potential issues to be
% identified
% and corrected.

% 1. In a first phase, starting ** Tuesday, Feb 24th 2015 at 09:00 **
% Part of the Gaia cluster (gaia-83 to gaia-154) will be migrated to Debian 7,
% during a maintenance which will last until 18:00 on the mentioned date,
% with
% Gaia unavailable for the duration.
% A new property will be set in the OAR scheduler that will allow you to
% target
% either the migrated nodes or the old nodes (technical details below).
% You are strongly encouraged to test that your workflow is working correctly
% on the migrated nodes or take the appropriate corrective measures
% (recompile
% software, etc.) as warranted.
% This initial testing phase is meant to last 3 weeks.

% 2. Second phase, starting (tentatively) ** Tuesday, Mar 17th 2015 at 09:00
% **
% The rest of the Gaia cluster (gaia-1 to gaia-79) and the Chaos cluster
% nodes
% will be migrated upon the successful conclusion of the trial Phase-1.
% Both Gaia and Chaos will be unavailable until 18:00.

% 3. Gaia storage reorganization: ** Tuesday, Mar 24th 2015 at 09:00 **
% The home directories of Gaia will be moved from the current NFS server to
% a distributed GPFS [*] setup which will allow for better stability, speed,
% and future storage extensions. A half-day maintenance will be performed on
% Gaia, with the cluster unavailable until 14:00.



% %=======================================================
% \section{\faCommentAlt\ UL HPC Highlights}
% %=======================================================

% Similar to an executive summary of (incoming/passed)
% milestones (half page)

% %=======================================================
\section{\faBullhorn\ What's new?}

\input{_gpfs.md}
%\input{}

% %=======================================================

% New features: services/improvements we've introduced, new software
% available, etc.

% %=======================================================
% \section{\faBarChart\ Latest Statistics}
% %=======================================================

% Platform overview (pretty graphs): for the passing \$period (maybe also
% with last year history): \#jobs , data transferred,  new users/projects,
% etc., also reminder of existing services (gitlab, gforge, owncloud, etc.)

% %=======================================================
% \section{\faShare\ Outcomes}
% %=======================================================

% Outcomes: projects/groups using the clusters, new articles published
% referencing the platform, etc.

% %=======================================================
% \section{\faUser\ User Focus}
% %=======================================================

% User stories: 1-2 paragraphs from (1-2) 'big' users, describing what
% they're doing

% %=======================================================
% \section{\faCalendar\ Upcoming Events}
% %=======================================================
% Upcoming events: new tutorials, software we'll introduce in the next
% \$period, upcoming HPC schools etc.

% %============================================================================================
% \section{\faCogs\ How to use the \texttt{ulhpc\_newsletter1} style?}
% %============================================================================================

% \faWarningSign\ : You need to compile the newsletters with \texttt{lualatex}! 

% If you are using \href{https://github.com/Falkor/Makefiles/blob/devel/latex/Makefile}{my generic Makefile}, simply place a file \texttt{.Makefile.local} at the root of your document having the following content:

% \lstinputlisting[language=make]{.Makefile.local}

% Then in addition to the addition of the \LaTeX\ style file \texttt{ulhpc\_newsletter1.sty} at the root of your document, you'll need: 

% \begin{itemize}
%   \item a copy of the logos files (in PDF format) \texttt{logo\_UL.pdf} and \texttt{logo\_ULHPC.pdf} in a know path;
%   \item to initiate a certain number of variables / macros used in the style \textbf{prior to} the call to the style package.
% \end{itemize}

% They are defined as follows: 

% \lstinputlisting[language=TeX,firstline=45,lastline=70]{ulhpc_newsletter1.sty}


% % You need first to clone (directly or as a git submodule) the \texttt{ULHPC/documents} repository on \href{https://github.com/ULHPC/documents}{Github}. 

% % Then in addition to the addition of the \LaTeX\ style file \texttt{ulhpc\_book.sty} at the root of your document, you'll need: 

% %=======================================================
% \section{\faBeer\ Last but not least}
% %=======================================================
% \leftinfo{Another highlight}{For this page... }

% \lipsum[2-8]

\end{document}



%~~~~~~~~~~~~~~~~~~~~~~~~~~~~~~~~~~~~~~~~~~~~~~~~~~~~~~~~~~~~~~~~
% eof
%
% Local Variables:
% mode: latex
% mode: flyspell
% mode: visual-line
% TeX-master: "ulhpc_newsletter-issue-01"
% End:
