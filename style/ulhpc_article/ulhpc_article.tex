%=============================================================================
% File:  ulhpc_article.tex --  Example of usage of the style
% Time-stamp: <Lun 2015-03-02 18:22 svarrette>
%=============================================================================
\documentclass{article}

\usepackage{graphicx}
\graphicspath{{logos/}} % add logos/ to the search path for images

\def\draft{} % comment to unset draft mode
%\def\showversion{no} % uncomment to remove the version printing 

\usepackage{ulhpc_article}
\def\AuthorPrefix{Management Team}
\def\frontabstract{
  This is the abstract to be place on the front-page using the macro \texttt{$\backslash$frontabstract\{...\}}
}

\title{The Title}
\def\subtitle{The subtitle}
\author{%
  Prof. Pascal Bouvry and\\
  Dr. S\'ebastien Varrette
}

% \date{
%   February 2015
% }

\begin{document}

\coverpages

\section{Section 1}

\lipsum[1]

\subsection{subsections}

\lipsum[2]

\section{Section 2}

\begin{tcolorbox}
  \lipsum[3]
\end{tcolorbox}

\newpage
%====================
\section{How to use the \texttt{ulhpc\_article} \LaTeX\ style?}

\begin{tcolorbox}[colback=red!5!white]
\faWarningSign\ : You need to compile the articles with the \texttt{lualatex} compiler for \LaTeX!  
\end{tcolorbox}
This is required by the \texttt{fontawesome} and some TikZ fonts. 
If you are using \href{https://github.com/Falkor/Makefiles/blob/devel/latex/Makefile}{my generic Makefile}, 
simply place a file \texttt{.Makefile.local} at the root of your document having the following content:

\lstinputlisting[language=make]{.Makefile.local}

\noindent Then in addition to the addition of the \LaTeX\ style file \texttt{ulhpc\_article.sty} at the root of your document, you'll need: 
\begin{itemize}
  \item a copy of the logos files (in PDF format) \texttt{logo\_UL.pdf} and \texttt{logo\_ULHPC.pdf} in a know path;
  \item to initiate a certain number of variables / macros used in the style \textbf{prior to} the call to the style package.
\end{itemize}

\noindent They are defined as follows: 

\lstinputlisting[language=TeX,firstline=44,lastline=70]{ulhpc_article.sty}

For instance, this is how this style is used to generate this document: 

\lstinputlisting[language=TeX,firstline=5,lastline=76]{ulhpc_article.tex}

\end{document}

%~~~~~~~~~~~~~~~~~~~~~~~~~~~~~~~~~~~~~~~~~~~~~~~~~~~~~~~~~~~~~~~~
% eof
%
% Local Variables:
% mode: latex
% mode: flyspell
% mode: visual-line
% TeX-master: "ulhpc_article"
% End:
